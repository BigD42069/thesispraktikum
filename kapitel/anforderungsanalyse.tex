\chapter{Analyse der Anforderungen}
Um eine funktionale, sichere und wirtschaftlich sinnvolle Lösung zu entwickeln, ist vorab eine strukturierte Analyse der projektbezogenen Anforderungen notwendig. Diese lassen sich in funktionale, nicht-funktionale, technische sowie sicherheitsrelevante Anforderungen gliedern. Darüber hinaus wird ein kurzer Vergleich zu bestehenden Lösungen vorgenommen, um die Positionierung des geplanten Systems klar abzugrenzen.
\section{Funktionale Anforderungen}
Die zentralen Funktionen des zu entwickelnden Auslesegeräts ergeben sich direkt aus den gesetzlichen Vorgaben sowie aus den praktischen Anforderungen von Transportunternehmen. Im Einzelnen ergeben sich folgende funktionale Anforderungen:
	•	Auslesen von Tachographendaten gemäß den EU-Verordnungen 2016/799 und 2021/1228
	•	Unterstützung der seriellen Kommunikationsprotokolle des Tachographen (UART, K-Line)
	•	Sichere Authentifizierung über Unternehmenskarte oder vorab registrierte Geräteidentitäten
	•	Datenverschlüsselung während der Übertragung und ggf. Speicherung
	•	Bereitstellung der Daten zur Anzeige, Archivierung oder Export
	•	Plattformunabhängiger Datenzugriff, z. B. über Webanwendung oder mobile App
	•	Energieeffizienter, mobiler Betrieb ohne stationäre Infrastruktur
Diese Funktionen bilden den sogenannten „Minimal Viable Product“-Kern des Systems und müssen prioritär umgesetzt werden.
\section{Nicht-funktionale Anforderungen (Mobilität, Benutzerfreundlichkeit, Kosten)}
Nicht-funktionale Anforderungen betreffen qualitative Merkmale des Systems, die nicht direkt durch eine einzelne technische Funktion abgedeckt sind, aber für die Akzeptanz und den praktischen Einsatz entscheidend sind:
	•	Mobilität: Kompakte Baugröße, Batteriebetrieb möglich
	•	Benutzerfreundlichkeit: Einfache Bedienung auch für technisch nicht geschulte Anwender
	•	Robustheit: Geeignet für den Außeneinsatz im Fahrzeugumfeld
	•	Kosteneffizienz: Komponentenwahl abgestimmt auf niedrige Materialkosten
	•	Modularität: Erweiterbarkeit durch Software oder zusätzliche Schnittstellen
	•	Zukunftsfähigkeit: Berücksichtigung künftiger Tachograph-Generationen (z. B. Smart Tachographs Version 2)
\section{Sicherheits- und Datenschutzanforderungen}
Die Arbeit mit personenbezogenen und unternehmenskritischen Daten erfordert besondere Vorkehrungen im Design:
	•	Vertraulichkeit: Schutz der ausgelesenen Daten vor unbefugtem Zugriff
	•	Integrität: Sicherstellung, dass Daten vollständig und unverändert übertragen werden
	•	Authentifizierung/Autorisierung: Zugriff nur durch berechtigte Nutzer oder Systeme
	•	Speichersicherheit: Daten sollen entweder temporär im Gerät oder sicher extern (z. B. Cloud) abgelegt werden – Speicherung im Klartext im Gerät ist zu vermeiden
	•	Datenschutzrechtliche Konformität: Umsetzung gemäß DSGVO und IT-Sicherheitsstandards
Insbesondere TLS/SSL-Verschlüsselung, tokenbasierte Autorisierung und rollenbezogene Zugriffskontrolle (z. B. Fahrer vs. Unternehmen) werden als technische Mittel berücksichtigt.
\section{Technische Spezifikationen gemäß Normen}
Die Kommunikation mit dem digitalen Tachographen muss normenkonform erfolgen. Relevante technische Richtlinien sind u. a.:
	•	ISO 16844 (Teil 1–6): definiert die physikalischen und Protokoll-Schichten für Tachographenschnittstellen
	•	ISO 9141-2: Kommunikationsprotokoll für K-Line-Diagnoseschnittstellen
	•	Continental VDO Datenformate (z. B. DDD/XML-Files): strukturierte Fahrerdatenformate
	•	EU-Verordnung 2016/799, Anhang 1C: Vorgaben zu Authentifizierungsprozessen, Protokollformaten und Sicherheitsmechanismen
Die Umsetzung muss kompatibel mit diesen Spezifikationen erfolgen, um rechtssichere Daten zu gewährleisten.
\section{Vergleich zu bestehenden Lösungen}
Der Markt für mobile Tachographen-Auslesesysteme ist zwar vorhanden, jedoch meist auf professionelle Großlösungen oder fest verbaute Telematikeinheiten beschränkt. Kritische Einschränkungen der am Markt befindlichen Lösungen:
\includegraphics[width=0.2\linewidth]{bilder/table.png}