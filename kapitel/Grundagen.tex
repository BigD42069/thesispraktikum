
\chapter{Grundlagen}

\section{Rechtlicher Rahmen digitaler Tachographen}
Der Einsatz digitaler Tachographen in Nutzfahrzeugen innerhalb der Europäischen Union unterliegt strengen rechtlichen Vorgaben. 

\subsection{EU-Verordnung 2016/799}
Grundlage bildet die Durchführungsverordnung (EU) 2016/799, die die technischen Vorschriften für digitale Kontrollgeräte gemäß der Verordnung (EG) Nr. 561/2006 konkretisiert. Ziel ist es, die Einhaltung von Lenk- und Ruhezeiten, die Verkehrssicherheit sowie den fairen Wettbewerb im Güter- und Personenverkehr zu gewährleisten.
\subsection{EU-Verordnung 2021/1228}
Mit der Einführung der Durchführungsverordnung (EU) 2021/1228 wurden ergänzende Anforderungen an intelligente Tachographen (Smart Tachographs) der zweiten Generation definiert, insbesondere im Hinblick auf Konnektivität, Fernabfrage und verschärfte Sicherheitsmechanismen. Unternehmen sind verpflichtet, die aufgezeichneten Daten regelmäßig auszulesen, zu archivieren und im Streitfall bereitstellen zu können. Diese gesetzlichen Anforderungen bilden den funktionalen Rahmen für das hier entwickelte Auslesegerät.

\section{Funktionsweise und Datenstrukturen digitaler Tachographen}
Ein digitaler Tachograph besteht im Wesentlichen aus drei Komponenten: dem Bordgerät, der Fahrzeug-Einheit sowie den Fahrerkarten. Das Bordgerät zeichnet kontinuierlich Informationen über Geschwindigkeit, Lenk- und Ruhezeiten, Ereignisse (z. B. Geschwindigkeitsüberschreitungen) und Positionsdaten auf. Die Daten werden sowohl auf der internen Speichereinheit des Geräts als auch auf den smartcard-basierten Fahrerkarten abgelegt.
Die Kommunikation erfolgt über eine serielle Datenschnittstelle, wobei spezifische Datenformate gemäß dem Continental VDO Format oder in XML/DDD-Strukturen genutzt werden. Für das Auslesen sind Zugriffsrechte auf Basis sogenannter Unternehmens- oder Kontrollkarten erforderlich. Daraus leitet sich ein erhöhtes Bedürfnis nach Datenintegrität, Verschlüsselung und Authentifizierung ab, was bei der Entwicklung eigener Auslesegeräte berücksichtigt werden muss.

\section{Kommunikationsschnittstellen von Tachographen}
Moderne Tachographen unterstützen mehrere physikalische Kommunikationsschnittstellen. Besonders relevant im Kontext dieser Arbeit sind:
\subsection{UART}
	UART (Universal Asynchronous Receiver/Transmitter): UART Kommunikation oder Universal Asynchronous Receiver Transmitter bezeichnet ein grundlegendes Verfahren der seriellen Datenübertragung, das auf Hardwareebene implementiert ist und in nahezu allen Mikrocontrollern, Prozessoren und vielen elektronischen Systemen integriert wird. Sie arbeitet asynchron, das heißt ohne gemeinsame Taktleitung zwischen Sender und Empfänger, und setzt stattdessen eine genaue zeitliche Synchronisation über eine zuvor vereinbarte Baudrate voraus. Diese Baudrate ist die Anzahl der Symbole pro Sekunde und definiert die Bitzeit. Übliche Baudraten sind 300, 1200, 9600, 19200, 38400, 57600 oder 115200 Bits pro Sekunde. Moderne Systeme können deutlich höhere Raten wie 1 Megabit pro Sekunde und mehr erreichen, erfordern dafür jedoch präzisere Taktgeber.
Die UART Schnittstelle ist auf Hardwareebene als Transceiver implementiert und besteht aus mehreren Hauptkomponenten. Dazu zählen ein Baudratengenerator, ein Sender bestehend aus Puffer und Schieberegister, ein Empfänger mit synchronisiertem Schieberegister und optionalen FIFO Puffern sowie eine Steuer und Statuslogik. Der Baudratengenerator ist ein programmierbarer Teiler, der aus einem höheren Systemtakt das Bitzeit Signal erzeugt. In klassischen Implementierungen wie dem 16550 UART wird eine Basisfrequenz, oft 1,8432 Megahertz, durch einen programmierbaren Teilerfaktor geteilt, um die gewünschte Baudrate zu erreichen.
Der Sender nimmt Daten typischerweise in einem parallelen Datenformat vom Prozessor entgegen und speichert sie in einem Sendepuffer oder einer FIFO Warteschlange. Von dort werden die Daten seriell in das Sendeschieberegister geladen. Die Kontrolllogik fügt automatisch ein Startbit, optional ein Paritätsbit und ein oder mehrere Stoppbits hinzu. Die Übertragung beginnt stets mit einem Startbit, das durch das Setzen der Leitung auf logisch Null für die Dauer einer Bitzeit signalisiert wird. Anschließend werden die Datenbits, beginnend mit dem niederwertigsten Bit, seriell übertragen. Nach den Datenbits folgt optional ein Paritätsbit, das eine einfache Fehlererkennung ermöglicht, und schließlich ein oder zwei Stoppbits, die durch das Setzen der Leitung auf logisch Eins für eine oder zwei Bitzeiten gekennzeichnet sind. Der Sender stellt den Datenstrom kontinuierlich bereit, solange Daten im FIFO Speicher liegen.
Der Empfänger arbeitet komplementär. Er überwacht die Empfangsleitung und erkennt ein Startbit durch einen Wechsel von logisch Eins in den Ruhezustand auf logisch Null. Nach der Erkennung des Startbits aktiviert der Baudratengenerator die Abtastung des Signals in festen Zeitintervallen. Viele moderne UART Implementierungen verwenden ein Oversampling Verfahren, bei dem jedes Bit mehrfach abgetastet wird, typischerweise sechzehnmal, um eine präzise Synchronisation und robuste Bitentscheidung zu gewährleisten. Das Empfängerschieberegister sammelt die Bits und stellt sie nach vollständigem Empfang eines Datenrahmens als parallelen Wert bereit. Nach erfolgreichem Empfang kann die Hardware Paritätsfehler oder Framingfehler melden, falls das Paritätsbit nicht mit der erwarteten Parität übereinstimmt oder wenn das Stoppbit nicht korrekt erkannt wird.
Auf Registerebene besteht eine klassische UART Einheit aus mehreren Steuer und Statusregistern. Ein typisches Design wie der 16550 UART enthält ein Transmit Holding Register zum Schreiben von zu sendenden Daten und ein Receive Holding Register zum Lesen empfangener Daten. Ein Line Control Register definiert die Anzahl der Datenbits, Paritätseinstellungen und die Anzahl der Stoppbits. Ein Baudrate Divisor Register steuert den Teiler des Baudratengenerators. Ein Interrupt Enable Register und ein Interrupt Identification Register erlauben eine ereignisgesteuerte Datenübertragung, indem Interrupts ausgelöst werden, wenn ein Zeichen empfangen wurde, der FIFO Speicher leer ist oder Fehler erkannt wurden. Moderne Implementierungen enthalten zusätzliche FIFO Register mit typischerweise sechzehn Byte Tiefe, wodurch der Prozessor weniger häufig Interrupts verarbeiten muss und die Effizienz bei hohen Baudraten gesteigert wird.
Die physikalische Signalisierung kann in verschiedenen Standards erfolgen. Bei direkter Verbindung von Mikrocontrollern wird UART üblicherweise mit TTL Pegeln betrieben, typischerweise 0 Volt für logisch Null und 3,3 Volt oder 5 Volt für logisch Eins. Für robustere Anwendungen wird UART häufig mit Schnittstellenstandards wie RS232, RS422 oder RS485 kombiniert. RS232 invertiert die Signalpegel und verwendet Spannungen von plus oder minus zwölf Volt, wodurch eine höhere Störsicherheit erreicht wird. RS422 und RS485 implementieren differenzielle Signalübertragung über verdrillte Leitungen, wodurch größere Reichweiten und Datenraten bis zu mehreren Megabit pro Sekunde möglich sind.
UART Kommunikation ist paketlos und ohne eingebaute Adressierungslogik. Daher wird sie oft als Grundlage für höhere Protokolle genutzt. Beispiele sind Modbus RTU, DMX512 in der Veranstaltungstechnik oder proprietäre Protokolle in Industrie und Automatisierung. In diesen Fällen dient UART als physikalische Transportschicht, während Fehlerkorrektur, Adressierung und Steuerung auf einer höheren Ebene implementiert werden.
Auf Systemebene ist UART durch seine Einfachheit und Robustheit ein unverzichtbares Kommunikationsmittel. In Mikrocontrolleranwendungen wird sie häufig als Standard Debug Schnittstelle verwendet. Auch Bootloader Implementierungen nutzen UART aufgrund der geringen Komplexität und der Möglichkeit, Geräte mit minimaler Hardwareunterstützung zu programmieren. Trotz der Verfügbarkeit schnellerer serieller Schnittstellen wie SPI, I2C oder USB bleibt UART weit verbreitet, da sie für Punkt zu Punkt Verbindungen mit geringen Hardwareanforderungen optimiert ist.
Timingdiagramme einer UART Übertragung zeigen den Wechsel der Leitung von Ruhezustand auf Startbit, gefolgt von den Datenbits in Little Endian Reihenfolge, optionalem Paritätsbit und Stoppbit. Der Empfänger synchronisiert sich auf die Flanke des Startbits und liest die Bits in der Mitte jeder Bitzeit aus. Eine präzise Taktauslegung ist entscheidend, da Toleranzen in der Baudrate kumulative Fehler verursachen können. Moderne Systeme spezifizieren eine maximale Baudratendifferenz von zwei bis fünf Prozent zwischen Sender und Empfänger.
Zusammenfassend ist UART Kommunikation ein elementares, hardwareunterstütztes Verfahren für serielle Punkt zu Punkt Datenübertragung, das aus Sicht von Hardwarearchitektur, Timing und Registerebene gut definiert ist. Ihre Einfachheit, Flexibilität und universelle Unterstützung machen sie zu einem Basiselement digitaler Systeme, das in nahezu allen Bereichen von eingebetteter Elektronik, industrieller Automatisierung, Kommunikationsinfrastruktur und PC Peripherie Anwendung findet.
\subsection{K-Line}
	K Line Kommunikation ist ein standardisiertes serielles Kommunikationsverfahren, das in der Automobilindustrie insbesondere für On Board Diagnose Systeme verwendet wird. Sie ist in mehreren internationalen Normen spezifiziert, darunter ISO 9141 und ISO 14230, auch bekannt als Keyword Protocol 2000. Die K Line dient als bidirektionale serielle Datenleitung zur Übertragung von Diagnosedaten zwischen Steuergeräten im Fahrzeug und externen Diagnosegeräten. Ihre Einführung diente der Vereinheitlichung von Diagnoseschnittstellen, um Herstellern, Werkstätten und Prüfstellen den Zugriff auf fahrzeuginterne Informationen zu ermöglichen.
Die K Line arbeitet als halbduplexe serielle Schnittstelle, das heißt, dass Senden und Empfangen über dieselbe Leitung erfolgt, jedoch nicht gleichzeitig. Die physikalische Signalübertragung erfolgt typischerweise mit einer Ruhespannung von etwa zwölf Volt, die bei aktiver Kommunikation auf Null Volt gezogen wird. Dadurch kann die K Line direkt in Fahrzeugsystemen mit Bordnetzspannungen von zwölf oder vierundzwanzig Volt eingesetzt werden. Eine optionale zweite Leitung, die sogenannte L Line, wurde in älteren Systemen für Initialisierungszwecke genutzt, ist jedoch in modernen Implementierungen entfallen.
Auf Protokollebene basiert die K Line auf asynchroner serieller Kommunikation ähnlich wie bei UART. Die Daten werden in Frames mit einem Startbit, acht Datenbits, optionalem Paritätsbit und einem Stoppbit übertragen. Die Baudrate liegt typischerweise bei 10400 Bits pro Sekunde, kann aber je nach Fahrzeug und Norm variieren. ISO 9141 beschreibt ein festes Kommunikationsschema, während ISO 14230 zusätzliche Flexibilität durch verschiedene Initialisierungsverfahren und ein erweitertes Protokolldesign bietet.
Die Initialisierung der Kommunikation kann auf zwei Arten erfolgen. Bei der sogenannten Fast Initialization wird die Leitung kurzzeitig auf logisch Null gezogen, um dem Steuergerät das Kommunikationsinteresse zu signalisieren, woraufhin mit fester Baudrate Daten gesendet werden. Bei der Five Baud Initialization sendet das Diagnosegerät zunächst eine fünf Bit lange Adressnachricht mit einer Geschwindigkeit von fünf Baud, also einem Bit alle zweihundert Millisekunden. Dieses Verfahren ermöglicht es älteren Steuergeräten, ohne dedizierte Hardware auf eine standardisierte Weise angesprochen zu werden. Nach erfolgreicher Initialisierung wird die Kommunikation mit der festgelegten Baudrate fortgesetzt.
Der physikalische Aufbau einer K Line Schnittstelle erfolgt meist über einen Pull Up Widerstand von etwa 510 Ohm zur Bordnetzspannung, sodass die Leitung im Ruhezustand auf hohem Pegel liegt. Aktive Treiber ziehen die Leitung auf Null Volt, um logische Null zu signalisieren. Aufgrund der elektrischen Eigenschaften und der Leitungsführung in Fahrzeugen ist die K Line relativ robust gegenüber Störungen, arbeitet jedoch nur als Punkt zu Punkt Verbindung zwischen einem Steuergerät und einem Diagnosegerät. Mehrere Steuergeräte im Fahrzeug können über eine gemeinsame K Line verbunden sein, jedoch darf immer nur ein Steuergerät aktiv antworten, während alle anderen ihre Ausgangsstufe deaktivieren. Dieses Konzept wird als Open Collector oder Open Drain Ansteuerung bezeichnet und ermöglicht eine einfache Busstruktur ohne komplexe Kollisionserkennung.
Auf Protokollebene unterstützt die K Line verschiedene Rahmenformate. ISO 9141 definiert einen einfachen Aufbau mit einem Startbyte, einem Steuerbyte, einem Datenfeld und einer Prüfsumme. ISO 14230 erweitert dieses Konzept um ein flexibles Headerformat mit Längenangaben, Zieladressen und Quelladressen, wodurch eine strukturierte Kommunikation mit mehreren Steuergeräten möglich ist. Die Prüfsumme wird meist als einfache Addition aller Bytes berechnet und modulo 256 übertragen. Dieses Verfahren bietet grundlegende Fehlererkennung, jedoch keine Fehlerkorrektur.
Hardwareseitig besteht eine K Line Schnittstelle typischerweise aus einem Treiberchip, der Pegelanpassung, Schutzfunktionen und Sende und Empfangslogik implementiert. Beispiele für solche Bausteine sind die ISO 9141 Transceiver Chips wie der L9637D oder der MC33290. Diese Chips übernehmen das Umschalten zwischen Sende und Empfangsmodus und bieten Schutzfunktionen gegen Kurzschlüsse, Überstrom und Überspannung, die in der automobilen Umgebung mit hohen Störpegeln notwendig sind.
Die K Line wurde über viele Jahre als Standard in der Fahrzeugdiagnose verwendet und ist ein zentraler Bestandteil der ersten und zweiten Generation von On Board Diagnose Systemen. Mit der Einführung von OBD II in den Vereinigten Staaten und EOBD in Europa wurde K Line Kommunikation flächendeckend implementiert. Sie ermöglicht den Zugriff auf Fehlercodes, Sensordaten und Steuergeräteinformationen. Ab etwa 2008 wurde sie zunehmend durch den CAN Bus verdrängt, bleibt jedoch in vielen Fahrzeugen als Diagnosezugang erhalten, insbesondere bei älteren Modellen.
Ein wesentlicher Vorteil der K Line liegt in ihrer Einfachheit und niedrigen Kosten. Sie erfordert nur eine Datenleitung und einen einfachen Transceiver Chip. Allerdings sind die Datenraten im Vergleich zu modernen Bussystemen wie CAN oder FlexRay begrenzt. Die Halbduplex Kommunikation und der Punkt zu Punkt Charakter machen K Line vor allem für Diagnosezwecke geeignet, während sicherheitskritische Systeme auf schnellere und deterministischere Busse angewiesen sind.
Zusammenfassend ist die K Line Kommunikation eine robuste, standardisierte serielle Schnittstelle, die durch ihre breite historische Verbreitung ein Schlüsselwerkzeug für die Diagnose und Wartung von Fahrzeugen darstellt. Sie kombiniert eine einfache elektrische Architektur mit definierten Protokollen nach ISO 9141 und ISO 14230 und bildet eine Brücke zwischen früheren fahrzeugspezifischen Diagnosesystemen und modernen On Board Diagnoseverfahren. Auch wenn sie heute durch CAN und andere Bussysteme ergänzt oder ersetzt wird, bleibt sie ein relevantes Element in der automobilen Elektronik und ein Beispiel für die Evolution standardisierter Schnittstellen in sicherheitskritischen technischen Umgebungen.


\section{Sicherheitsanforderungen (Authentifizierung, Autorisierung, Verschlüsselung)}
Da Tachographendaten sensible Informationen wie Lenkzeiten, Arbeitszeitprotokolle und Geschwindigkeitsverläufe enthalten, sind sie besonders schützenswert. Die Kommunikation mit Auslesegeräten unterliegt daher folgenden sicherheitsrelevanten Anforderungen:
	Authentifizierung: Nur autorisierte Geräte bzw. Karten dürfen Zugriff auf Tachograph-Daten erhalten. Dies wird durch Challenge-Response-Mechanismen und Sicherheitszertifikate sichergestellt.
	Autorisierung: Der Zugriff darf nur auf jene Daten gewährt werden, die im Rahmen des jeweiligen Benutzerprofils rechtlich erlaubt sind (Fahrer, Unternehmen, Kontrolle).
	Verschlüsselung: Die Datenübertragung – insbesondere bei drahtloser Kommunikation – erfolgt vorzugsweise verschlüsselt, häufig unter Nutzung von TLS (Transport Layer Security).
Diese Punkte stellen hohe Anforderungen an das Firmware- und Kommunikationsdesign und waren zentraler Bestandteil im Entwicklungskonzept des hier vorgestellten Prototyps.

\section{Überblick Mikrocontroller-Plattform ESP32}
Die hardwareseitige Realisierung des Auslesegeräts basiert maßgeblich auf dem ESP32, einem leistungsfähigen Mikrocontroller von Espressif Systems.
\subsection{Architektur und Leistungsdaten}
Der ESP32 ist ein 32-Bit-Mikrocontroller mit zwei CPU-Kernen, arbeitet mit einer Taktfrequenz von bis zu 240 MHz und bietet je nach Modell bis zu 520 KB SRAM sowie mehrere Megabyte Flash-Speicher. Neben klassischen IO-Schnittstellen (GPIO, ADC, SPI, I²C) überzeugt er insbesondere durch die Integration von Wi-Fi (802.11 b/g/n) und Bluetooth (Classic  \& BLE), was ihn für smarte Embedded-Projekte – wie dieses – besonders attraktiv macht.
\subsection{Wi-Fi}
	Wi-Fi dient der drahtlosen Verbindung zu Cloud-Systemen oder mobilen Endgeräten (z. B. dem Smartphone des Flottenbetreibers).
\subsection{Bluetooth low energy}
Bluetooth Low Energy (BLE) ist eine speziell für äußerst energieeffiziente Anwendungen entwickelte Funktechnologie, die im 2{,}4\,GHz ISM-Band (2{,}402--2{,}480\,GHz) arbeitet und dabei 40 Kanäle mit jeweils 2\,MHz Kanalabstand nutzt, wovon drei als Werbekanäle und 37 als Datenkanäle definiert sind. Im Vergleich zu Bluetooth Classic, das 79 Kanäle mit 1\,MHz Abstand verwendet, zeichnet sich BLE durch eine optimierte Kanalnutzung und Frequenzsprungverfahren (Frequency-Hopping Spread Spectrum, FHSS) aus, um Interferenzen zu minimieren und die Energieeffizienz zu steigern. Die Modulation erfolgt über GFSK (Gaussian Frequency-Shift Keying), während Bluetooth Classic zusätzlich komplexere Modulationsschemata wie $\pi/4$-DQPSK oder 8DPSK verwendet. BLE bietet eine flexible Auswahl an physikalischen Datenraten: LE~2M~PHY mit 2\,Mbit/s, LE~1M~PHY mit 1\,Mbit/s sowie LE~Coded~PHY mit Datenraten von 500\,kbit/s (S=2) oder 125\,kbit/s (S=8), wodurch Reichweite und Robustheit der Verbindung anwendungsabhängig optimiert werden können. Die Empfindlichkeit der Empfänger reicht von $-70\,\mathrm{dBm}$ bei 2M und 1M~PHY bis hin zu $-82\,\mathrm{dBm}$ bei S=8, was eine hohe Reichweite auch bei geringer Sendeleistung ermöglicht. BLE unterstützt eine maximale Sendeleistung von $+20\,\mathrm{dBm}$ (100\,mW).
Neben klassischen Punkt-zu-Punkt-Verbindungen wie Piconets unterstützt BLE moderne Topologien wie Broadcast- und Mesh-Netzwerke, was es besonders attraktiv für IoT-Anwendungen und große Sensornetzwerke macht. Zudem hat sich BLE von einer reinen Kommunikationsschnittstelle zu einer Technologie für präzise Positionsbestimmung entwickelt. Über Verfahren wie RSSI-Messung, Channel Sounding und Richtungsbestimmung (Angle of Arrival/Angle of Departure, AoA/AoD) können Anwender Geräte nicht nur erkennen, sondern auch ihre Entfernung und Richtung bestimmen. Ergänzt wird dies durch Unterstützung verschiedener Datenübertragungsarten wie asynchroner, synchroner und isochroner Kommunikation, sowohl verbindungsorientiert als auch verbindungslos. Durch seine energiesparende Architektur, hohe Flexibilität und integrierte Positionierungsfunktionen ist BLE heute eine Schlüsseltechnologie für Wearables, Smart-Home- und Industrie-4.0-Anwendungen sowie für ortsbasierte Dienste mit hoher Genauigkeit.
Zusätzlich besitzt der ESP32 mehrere UART-Schnittstellen, wodurch separate Kommunikationskanäle für Datenübertragung und Kalibrierung/Pegelkontrolle gegenläufiger Signale realisiert werden können.

\section{Relevante Zusatzkomponenten}
Zur Ergänzung des ESP32 kommen im Projektaufbau mehrere unterstützende Hardwarekomponenten zum Einsatz:

\subsection{MAX3232 Pegelwandler}
Der MAX3232 ist ein vielseitiger, dualer RS-232-Line-Treiber und -Empfänger, der speziell für den Betrieb an einzelnen Spannungsversorgungen im Bereich von 3\,V bis 5{,}5\,V ausgelegt ist und dadurch eine hohe Flexibilität für verschiedenste Anwendungen bietet. Das IC integriert zwei Treiber und zwei Empfänger sowie eine Dual-Charge-Pump-Schaltung, die mit vier externen 0{,}1-$\mu$F-Kondensatoren arbeitet, um die zur Pegelanpassung nötigen höheren Spannungen zu generieren. Charakteristisch ist außerdem ein niedriger Stromverbrauch von typischen 300\,$\mu$A, wodurch das Bauteil auch für battery-powered systems besonders geeignet ist. Der MAX3232 erfüllt die Anforderungen der TIA/EIA-232-F- und ITU-V.28-Standards und bietet darüber hinaus eine robuste ESD-Schutz-Zertifizierung von $\pm$15\,kV (Human-Body-Model), was ihn sehr widerstandsfähig gegen elektrostatische Entladungen macht. Technisch kann er Datenübertragungsraten bis zu 250\,kbit/s bewältigen und die Treiber weisen eine maximale Slew-Rate von 30\,V/$\mu$s auf. Ein weiteres nützliches Feature ist die Fähigkeit, 5-V-Logikpegel auf den Eingängen mit einer Versorgung von 3{,}3\,V zu akzeptieren, was die Integration mit gängigen Mikrocontroller-Boards vereinfacht. Erhältlich ist der MAX3232 in verschiedenen Gehäusevarianten wie SOIC, SSOP oder TSSOP (jeweils 16 Pins), was eine einfache Integration in unterschiedlich platzierte Leiterplattendesigns erlaubt.

\subsection{L9637D K-Line Transceiver}
Der L9637D ist ein monolithischer Bus-Treiber, der speziell für den Einsatz in automobilen Diagnosesystemen mit ISO~9141-Kompatibilität entwickelt wurde. Er verarbeitet Versorgungsspannungen im weiten Bereich von 4{,}5\,V bis 36\,V (Transientes bis 40\,V erlaubt) und schützt gegen eine umgekehrte Versorgung bis zu $-24$\,V. Im Standby-Modus weist der Baustein einen äußerst geringen Stromverbrauch auf, besonders bei VCC~$\leq$~0{,}5\,V, und auch im ausgeschalteten Zustand liegt der typische Ruhestrom bei lediglich 120\,$\mu$A. Der TX-Eingang ist TTL-kompatibel, während die bidirektionale K-I/O-Leitung eine schwellenwertabhängige Eingangscharakteristik besitzt, die sich mit der Versorgungsspannung ändert. Ein integrierter Übertemperaturschutz schaltet selektiv die K-Leitung ab, falls kritische Temperaturen erreicht werden. Widerstandsfähigkeit und Störfestigkeit werden durch definierte Ausschaltzustände bei Unterspannung oder bei Unterbrechung von VS oder GND sowie durch kontrollierte Ausgangsflanken zur Minimierung elektromagnetischer Emissionen gewährleistet. Weiterhin bietet der Baustein Eingangsimpedanzabsicherung bei offenem Versorgungspotenzial und definiert klare Zustände für LO oder RX bei offenem LI- bzw. K-Eingang sowie K-Ausschaltverhalten bei offenem TX-Eingang. Die Ein- und Ausgänge (TX, RX, LO) verfügen über interne Pull-up-Widerstände zur weiteren Stabilisierung. Der Ausgangsstrom der K-Leitung ist auf etwa 60\,mA begrenzt, und die Schnittstelle zeichnet sich durch hohe Eingangsimpedanzen sowie EMI-optimiertes Design aus. Funktional erlaubt der Baustein bidirektionale Kommunikation über die Diagnoseleitung (K-Line) und verfügt über ausreichende Hysterese zur Unterdrückung von Leitungsspannungsspitzen. Schutzmechanismen umfassen Überlastung, Kurzschluss gegen GND oder VS sowie Über- und Unterspannung. Timing-Eigenschaften bei Versorgungsspannungen zwischen 9 und 16\,V zeigen typische Werte: maximal 50\,kHz Übertragungsfrequenz, 2\,$\mu$s Anstiegs- und Abfallzeiten (bei spezifiziertem Lastglied) sowie definierte Einschaltzeiten ($t_\mathrm{ON}$) von bis zu 17\,$\mu$s und Ausschaltzeiten ($t_\mathrm{OFF}$) bis 17\,$\mu$s. Die Ausgangsflanken sind so gesteuert, dass EMI gering bleibt. Temperaturbereichsfähigkeit reicht von $-40\,^\circ$C bis $+150\,^\circ$C, und das Bauteil ist im platzsparenden SO-8-Gehäuse erhältlich. Insgesamt bietet der L9637D eine robuste, energieeffiziente und gut geschützte Lösung für ISO-9141-basierte Fahrzeugnetzwerkschnittstellen.

\subsection{MP1584EN-LF-Z Step-Down Regler}
Der MP1584 ist ein leistungsstarker, hochfrequenter Step-Down-Schaltregler mit integriertem Hochspannungs-Power-MOSFET, der bis zu 3\,A Ausgangsstrom liefert und mittels Strommodusregelung eine schnelle Regelantwort sowie einfache Kompensation ermöglicht. Der Eingangsspannungsbereich ist mit 4{,}5\,V bis 28\,V äußerst breit und eignet sich damit ideal für vielfältige Anwendungen -- etwa in der Automobilindustrie oder bei batteriegespeisten Systemen -- wobei der Ruhestrom bei lediglich etwa 100\,$\mu$A liegt, was die Energieeffizienz im Leerlauf deutlich optimiert. Um hohe Umwandlungseffizienz über verschiedene Lastbereiche hinweg zu gewährleisten, wird die Schaltfrequenz bei geringer Last reduziert, wodurch Schalt- und Treiberverluste minimiert werden. Dieser Mechanismus verhindert zudem, dass der Stromfluss im Induktor beim Start instabil wird (``frequency foldback''), und in Kombination mit einem thermischen Abschaltschutz sorgt er für eine zuverlässige und ausfallsichere Betriebsweise.
Mit einer programmierten Schaltfrequenz von bis zu 1{,}5\,MHz werden elektromagnetische Störungen, insbesondere im Bereich von AM-Radio oder ADSL, vermieden. Der Regler verfügt über eine integrierte Soft-Start-Funktion und eine präzise Strombegrenzung, die ohne externen Shunt-Widerstand auskommt. Die Schaltfrequenz lässt sich flexibel zwischen 100\,kHz und 1{,}5\,MHz per externem Widerstand einstellen. Das Bauteil ist in einem thermisch verbesserten SOIC-8E-Gehäuse mit Exposed Pad erhältlich, was zur effektiven Wärmeableitung beiträgt.
Die elektrischen Eigenschaften beinhalten eine genaue Referenz-Feedback-Spannung von ca.~0{,}8\,V, eine MOSFET-On-Resistance von etwa 150\,m$\Omega$, sowie eine Strombegrenzung um 4--4{,}7\,A. Weitere Merkmale sind ein UVLO-Startschwellenwert bei ungefähr 3\,V, ein Soft-Start-Intervall von ca.~1{,}5\,ms, minimale On- und Off-Zeiten von je 100\,ns, ein Shutdown-Strom von rund 20\,$\mu$A, sowie thermischen Schutz bei etwa 150\,$^\circ$C mit einer Hysterese von 15\,K. Der Enable-Pin lässt sich mit etwa 1{,}5\,V aktivieren bzw. bei 1{,}2\,V deaktivieren, über den Feedback-Pin wird die Ausgangsspannung über einen Widerstandsteiler präzise geregelt. Dabei beeinflusst die Einstellung der Frequenz am FREQ-Pin über einen externen Widerstand die Schaltfrequenz des Reglers.
Das MP1584-Modul kann Ausgangsspannungen im Bereich von 0{,}8\,V bis max.~25\,V erzeugen und ist stabil gegen Keramikkondensatoren. Typische Anwendungen umfassen Hochspannungskonversion, Automobil- und Industrie-Stromversorgungen sowie verteilte oder batteriegestützte Systeme. Die kompakte Bauform und thermische Optimierung machen dieses Bauteil zu einer effizienten und robusten Lösung für anspruchsvolle Gleichstrom-Versorgungsdesigns.
