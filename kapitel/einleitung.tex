\chapter{Einleitung}
\label{einleitung}

Einleitung in die Arbeit.
\section{Motivation und Problemstellung}
Die Beachtung von Fahr- und Pausenzeiten ist für Speditionen in der Europäischen Union nicht bloß eine Frage des Geschäfts, sondern vor allem eine rechtliche Pflicht, die eingehalten werden muss. Digitale Fahrtenschreiber sind dabei die wichtigsten Werkzeuge, um die nötigen Informationen über die Fahrten zu sammeln und aufzuzeichnen. Das Analysieren dieser Daten ist sehr wichtig, sowohl für Firmen (um selbst zu prüfen und zu speichern) als auch für die Behörden (um zu sehen, ob die Gesetze eingehalten werden).
In der Praxis zeigt sich jedoch, dass die bisher verfügbaren Lösungen zum Auslesen der Daten aus digitalen Fahrtenschreibern für viele Unternehmen sehr teuer sind. Außerdem sind die Geräte und Systeme oft so komplex, dass man sich intensiv damit auseinandersetzen muss, bevor man sie sicher und korrekt nutzen kann. Feste Anlagen sind in der Anschaffung besonders kostspielig, erfordern laufende Pflege und sind im Alltag häufig umständlich – vor allem dann, wenn LKWs nur selten zum Hauptsitz zurückkehren. Deshalb braucht man unbedingt mobile, günstige und einfach zu bedienende Lösungen, die trotzdem alle gesetzlichen Vorgaben erfüllen.

\section{Zielsetzung des Projekts}


\section{Abgrenzung und Schwerpunkt der Arbeit}
Die Studie taucht ein in die sonderbare Welt der direkten Gespräche mit digitalen Tachographen, geführt über spezielle Kabel (UART und K-Line nach ISO 9141 Art). Themen wie CAN-Bus-Gespräche, GNSS-Zeitreisen oder riesige Flottenzentralen werden mit Absicht ignoriert. Wir konzentrieren uns auch auf den Bau eines funktionierenden Modells, nicht aber auf eine riesige Fabrik oder einen offiziellen Verkaufsstart. Im Software-Dschungel beschränkt sich alles auf eine einfache Anzeige, die auf jedem Computer läuft und Daten exportieren kann. Das Erschaffen komplizierter Benutzerkonten, Wolkenverbindungen oder automatischer Berichte ist nicht Teil dieser Arbeit, wird aber kurz in der Zukunftsvision erwähnt.

\section{Aufbau der Arbeit}
Die Arbeit ist in acht Teile klar unterteilt. Nach dieser Einführung erklärt das zweite Kapitel die technischen und rechtlichen Dinge, die wichtig sind, um die spätere Ausführung zu verstehen. Kapitel drei konzentriert sich auf die Untersuchung dessen, was das geplante System können und technisch leisten muss. Danach, im vierten Kapitel, wird der technische Plan des Systems beschrieben – sowohl die Teile der Hardware als auch der Aufbau der Software. Kapitel fünf kümmert sich um die tatsächliche Ausführung, besonders den Bau des Prototyps und das Einsetzen der Kommunikationsabläufe sowie der Software. Kapitel sechs hält die durchgeführten Tests und ihre Resultate fest. In Kapitel sieben gibt es eine genaue Besprechung über das Erreichen der Ziele, die Arbeitsweise und mögliche Fehler des entwickelten Systems. Zum Schluss werden im achten Kapitel die wichtigsten Einsichten zusammengefasst und ein Blick auf mögliche neue Entwicklungen und Anwendungsmöglichkeiten geworfen.

